\section*{Problem definition and your task}

\paragraph{The graph coloring setup}
Fix an integer $n\ge 1$ and let $[n]=\{1,2,\dots,n\}$. Consider the complete graph $G$ on the vertex set $V(G)=[n]\times [n]$. For two distinct vertices
\[
v_1=(a_1,b_1),\qquad v_2=(a_2,b_2),
\]
the edge $e=v_1v_2$ is assigned one of four types:
\begin{itemize}
  \item \textbf{Type 1:} $(a_1-a_2)(b_1-b_2)>0$ (both coordinates move in the same direction; one point is strictly south--west of the other).
  \item \textbf{Type 2:} $(a_1-a_2)(b_1-b_2)<0$ (one coordinate increases while the other decreases).
  \item \textbf{Type 3:} $a_1=a_2$ (vertical edge).
  \item \textbf{Type 4:} $b_1=b_2$ (horizontal edge).
\end{itemize}
Define a color map $c:E(G)\to \mathcal{L}$ by
\[
c(e)=
\begin{cases}
\bigl(1,\ \min\{a_1,a_2\},\ \min\{b_1,b_2\}\bigr), & \text{if $e$ is type 1},\\[2pt]
\bigl(2,\ \min\{a_1,a_2\},\ \min\{b_1,b_2\}\bigr), & \text{if $e$ is type 2},\\[2pt]
\bigl(3,\ \min\{b_1,b_2\}\bigr), & \text{if $e$ is type 3},\\[2pt]
\bigl(4,\ \min\{a_1,a_2\}\bigr), & \text{if $e$ is type 4}.
\end{cases}
\]
For $U\subseteq V(G)$, let $G[U]$ denote the induced subgraph. An edge $e\in E(G[U])$ has a \emph{unique color in $U$} if no other edge of $G[U]$ has the same label $c(e)$.

\paragraph{Your task}
Show that for every vertex set $U\subseteq [n]\times[n]$ with $|U|\ge 2$, there exist $u_1,u_2\in U$ such that the edge $u_1u_2$ has a color that appears exactly once in $G[U]$ under the coloring $c$ above.

\section*{Summary of thoughts that might be useful}

\paragraph{Important note}
This is a summary of thoughts about the problem. Take it with a grain of salt. 

\paragraph{Main idea}
The key is to understand, for a fixed color label, how many edges of $G[U]$ can realize it. Two simple counting facts drive most of the structure.

\subsection*{Two local multiplicity facts}

\paragraph{Type 1.}
A color $(1,\alpha,\beta)$ can only come from edges whose south--west endpoint is exactly $(\alpha,\beta)$. Within $U$, its multiplicity equals the number of vertices of $U$ strictly north--east of $(\alpha,\beta)$ (and is $0$ if $(\alpha,\beta)\notin U$). In particular, a type--1 color is unique in $U$ if and only if some $p\in U$ has exactly one north--east neighbor in $U$.

\paragraph{Type 2.}
Fix $(\alpha,\beta)\in [n]^2$. Let
\[
r_U(\alpha,\beta)=\#\{(\alpha,y)\in U:\ y>\beta\},\qquad
s_U(\alpha,\beta)=\#\{(x,\beta)\in U:\ x>\alpha\}.
\]
Then the number of edges in $G[U]$ colored $(2,\alpha,\beta)$ is $r_U(\alpha,\beta)\cdot s_U(\alpha,\beta)$, since each such edge chooses one point above $(\alpha,\beta)$ in column $\alpha$ and one to the right in row $\beta$. Hence $(2,\alpha,\beta)$ is unique in $U$ if and only if $r_U(\alpha,\beta)=s_U(\alpha,\beta)=1$.

(Types 3 and 4 ignore one coordinate in their labels and therefore tend to repeat across columns/rows; the strategy below prefers to certify uniqueness via types 1 or 2.)

\subsection*{Immediate lemmas}

\paragraph{Lemma A (antichain $\Rightarrow$ unique type 2).}
If $U$ is an antichain under the product order on $\mathbb{Z}^2$ (no two points are coordinate-wise comparable), list $U$ in increasing $x$-order as $(x_1,y_1),\dots,(x_m,y_m)$. Then $y_1>\cdots>y_m$. For each consecutive pair $(x_i,y_i),(x_{i+1},y_{i+1})$, the edge has type 2 and color $(2,x_i,y_{i+1})$ with $r_U(x_i,y_{i+1})=s_U(x_i,y_{i+1})=1$, so that color appears exactly once.

\paragraph{Lemma B (a single NE neighbor $\Rightarrow$ unique type 1).}
If some $p\in U$ has exactly one north--east neighbor in $U$, then by the type--1 multiplicity fact the corresponding type--1 color appears exactly once.

These two lemmas already resolve some classes of $U$.
